\clearpage
\chapter{Conclusion}
Our aim in this thesis was to provide both an introduction to persistent homology as well as providing two examples of persistent homology being used in the wild in order to facilitate analysis of data.

We start by giving an introduction to homology, but focusing solely on simplicial homology while ignoring the more classical treatments which involve singular homology. Since simplicial homology can be considered conceptually more simple than other ways of defining homology this also leads to intuitive, immediate expressions of homology in terms of holes in simplicial complexes.

We then go on by defining persistent homology which gives us the tool with which we will perform our analysis. We explain

In the case of cornea 3D volumes gathered from microCT scans of the eyes of the species \textit{Bombus terrestris} we showed how persistent homology can be applied to volumetric data and how it can be used to perform a clustering and similarity analysis. In the case of synthetic networks from the striatrum we showed how persistent homology can be performed on graphs, in particular on directed graphs which contain assymetrical information about the network.

Using persistent homology we were able to relate the already observed fact that individuals of the species \textit{Bombus terrestris} have differ in morphology when it comes to larger and smaller individuals. The fact this is done with topological methods mean that the correlation we found is likely non-trivial.

By using these non-traditional ways of exploring data we show that perhaps there is some merit to considering persistent homology as a way of enhancing a traditional data analysis.
%%% Local Variables:
%%% mode: latex
%%% TeX-master: "thesis.tex"
%%% End:
