\chapter{Introduction}
Ordinary statistical analysis and machine learning are often used tools to understand and explore the increasing amounts of data that are present in the modern digital landscape. While these approaches continue to see great success, there is perhaps some value in exploring other avenues in mathematics that could prove useful in understanding data.

Topological data analysis is a form of data analysis that instead relies on topological information. There are several techniques used such as Mapper, Blah Blah, etc. The perhaps most applied one, and the one for which there exists proper software implementations such as (cite ripser etc), is persistent homology. It provides a way of quantifying and measuring the global shape of the data. While homology initially might be seen as something esoteric relegated to the realms of homological algebra, attempts have been made to use it as a tool for understanding data. Persistent homology is coarse enough to withstand noise that is often present in data (cite), while at the same time sophisticated enough to capture features which are particular to that dataset (cite). Homology being a functor on modules becomes simple linear algebra when working over fields, which immensly helps in the computational department.

The basic principle of persistent homology actually quite intuitive. We impose a simplicial complex on the dataset, that in some suitable sense should approximate a reasonable underlying topology in which the dataset lives, and then we compute the homology of this space. However, since there are many ways of approximating a simplicial complex on a set of points we consider not only one simplicial complex but rather a filtration of simplicial complexes parametrized by a given distance.

While the high-level idea is not very complicated, the devil is in the details. In order to rigorously define this notion as well as keep it flexible for other complexes than simplicial ones, we need to build a robust framework. This is done by defining a persistence complex as a sort of complex of chain complexes. Furthermore, we state the structure theorem for persistent homology which gives us an algebraic decomposition of persistent homology.

Our goal with this thesis is partly to provide an introduction to persistent homology as we would have liked it before we started this journey. As such, we have tried to keep a balance between the older material in the field that serves as foundational but also can be a bit outdate. Most of the definitions and results are accompanied by commentary, hopefully providing help along the way. We try to make the theoretical part somewhat self-contained, but some familarity with linear algebra, commutative algebra and module theory is needed.

The other part of the thesis, consists of two case studies. The case studies would not be possible without data provided by Zoological and Neurological. Our hope is that these two case studies although small show that persistent homology has potential as a tool in the toolbox of data analysis. We have taken care to conduct our analysis in such a way that we highlight how persistent homology enables the two approaches that we take.

We owe a lot to a variety of sources that are cited throughout the thesis. The main theory that we present was first developed by Zomordian in this form, although it was previously given in a more computational form by. The articles [] by Vejdemo provide a good overview and brief. Our novel contributions are our methodologies in the two case studies, although we it would not surprise us if similar approaches have been tried before in other domains.

 In the first case study, we analyze the corneas of the bumblebee \textit{Bombus terrestris}. By utilising persistent homology, we try to find differences between different-sized individuals of the species.

In the second case, we study a synthetic network generated based on the striatum, a part of the basal ganglia in the brain. There we attempt to see how the network compares to other models generated based on stochastic methods.



%%% Local Variables:
%%% mode: latex
%%% TeX-master: "thesis.tex"
%%% End:
