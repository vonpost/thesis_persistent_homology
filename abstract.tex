\newenvironment{abstract}%
    {\cleardoublepage\thispagestyle{empty}\null\vfill\begin{center}%
    \bfseries Abstract \end{center}}%
    {\vfill\null}
        \begin{abstract}
          Persistent homology is a way of giving a topological summary of a data-set. We give an introduction to the construction of persistent homology, including a proof of its algebraic decomposition as a graded module. This proof is constructive and yields an algorithm for computing persistent homology in a practical sense. In order to provide examples of the practical applications of persistent homology, we present two case studies. In the first case study we investigate the relationship between size and persistent homology of the bumblebee \textit{Bombus terrestris}. In the second case study we analyze the the persistent homology of a synthetic model of the striatum, a part of the basal ganglia in the brain.
        \end{abstract}
