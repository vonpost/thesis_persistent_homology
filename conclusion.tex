\clearpage
\chapter{Conclusion}


Our goal with this thesis is to provide both an introduction to theory of persistent homology, as well as examples of applications to real-world data. We believe this is  achieved.

We provide an exposition of persistent homology through the concept of a persistence module. We then state and finally prove the unique decomposition of persistence modules into a direct sum of free and torsional parts.  Furthermore, the proof of this theorem yields a concrete algorithm for computing persistent homology of a given filtration. By associating the decomposition with a barcodes, and further on persistence diagrams, we illustrate how persistent homology can be visualized and compared.

On the application side, we present two case studies. These studies are not to be seen as stand-alone results in their respective domains, but rather as examples of how persistent homology can be applied to achieve fruitful insights into data. By using these non-traditional ways of exploring data we hope that we show there is some merit to considering persistent homology as a way of enhancing a traditional data analysis.

In the first case study we analyze 3D scans of the corneas of the bumblebee \textit{Bombus terrestris}. This analysis shows how persistent homology can be applied to volumetric data and how it can be used to perform a clustering and similarity analysis. Furthermore, we are able to find that the persistent homology, specifically the barcode of $H_{2}$ compared across samples with the bottleneck distance, reinforces the already shown hypothesis in \cite{emily}, namely that the morphology of the eyes of \textit{Bombus terrestris} differs between smaller and larger individuals. We also interpret this result as implying that it is the density of the cornea which is a distinguishing factor between differently-sized individuals.

In the second case study we analyze a synthetically generated network made to mimic the microcircuitry of the striatum. By interpreting the network as a directed graph, we show how persistent homology can be used to compare real-world data given as graphs to control models generated in multiple ways. We reinforce the already established result in \cite{reimann}, that the resulting directed flag complex on the brain network displays a much richer simplicial structure in terms of dimensions and number of simplices compared to control models. Furthermore, we establish that $\beta_{1}$ of the microcircuitry is much lower than any of the control models. Finally, we see that the distribution of the persistent homology in $H_{1}$ of the microcircuitry is spread across the entire spectrum of possible degrees, whereas control models only have holes in small intervals of degrees. These observations could act as points of differentation when it comes to characterizing the striatum.

For some further directions we suggest .. in the Bombus. In the case of the synthetic microcircuitry of the striaum, an interesting lane of investigation is whether other filtrations than ones based on degree are as unique to the microcircuitry compared to the other models. Some other filtrations that can be used in the exact same methodology are other measures of importance of edges and vertices, such as edge-betweeness or clustering coefficients. Additionally, it could prove fruitful to see whether striatal microcircuitry \textit{in vivo} displays the same defining qualities in terms of persistent homology as the synthetical network.

%%% Local Variables:
%%% mode: latex
%%% TeX-master: "thesis.tex"
%%% End:
