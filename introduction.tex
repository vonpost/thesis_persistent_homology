\chapter{Introduction}
Using other measures than standard statistical analysis for datasets is an avenue that is explored in homology. From homology we can get a topological invariant of the data, interesting and can be further used as features in machine learning etc. Carlsson et al. describes it as exploring the shape of the data.

Persistent homology follows a basic principle that we can approximate a simplicial complex on a point cloud. This of course requires that the field is a metric space, because etc..

This thesis will serve as both an introduction the workings of persistent homology as well as (an?) example of persistent homology applied to a real dataset. Of course, since it is still quite a young field it is not entirely possible to answer to us what actual value persistent homology has as a tool within data science.

Etc etc.


%%% Local Variables:
%%% mode: latex
%%% TeX-master: "thesis.tex"
%%% End:
