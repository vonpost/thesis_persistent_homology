\newenvironment{abstract}%
    {\cleardoublepage\thispagestyle{empty}\null\vfill\begin{center}%
    \bfseries Abstract \end{center}}%
    {\vfill\null}
        \begin{abstract}
          Persistent homology is a way of giving a topological summary of a data-set. We give an introduction to   persistent homology, including a proof of its algebraic decomposition as a finitely generated graded module with respect to a polynomial ring over a field. This proof is constructive and yields an algorithm for computing persistent homology in a practical sense.

          In order to provide examples of the practical applications of persistent homology, we present two case studies. In the first case study we investigate the relationship between size and persistent homology of the bumblebee \textit{Bombus terrestris}. We find that the difference in size between samples is in some ways highly correlated to differences in persistent homology. In the second case study we analyze the persistent homology of a synthetic model of the striatum, a part of the basal ganglia in the brain. Here we find that the synthetic model differs in size and complexity from a number of control models when viewed through the lens of persistent homology and the accompanying theory. \end{abstract}
