\clearpage
\chapter{Conclusion}


Our goal with this thesis is to provide both an introduction to theory of persistent homology, as well as examples of applications to real-world data. We believe this is achieved.

We provide an exposition of persistent homology through the concept of a persistence module. We then state and finally prove the unique decomposition of persistence modules into a direct sum of free and torsional parts.  Furthermore, the proof of this theorem yields a concrete algorithm for computing persistent homology of a given filtration through the computation of graded Smith normal form. By associating the decomposition with barcodes, and further on persistence diagrams, we illustrate how persistent homology can be visualized. Furthermore, we review the bottleneck distance and $q$-Wasserstein distance which allows us to compare different barcodes with each other.

On the application side, we present two case studies. These studies are not to be seen as stand-alone results in their respective domains, but rather as examples of how persistent homology can be applied to achieve fruitful insights into data. By using these non-traditional ways of exploring data we hope that we show there is some merit to considering persistent homology as a way of enhancing a traditional data analysis.

In the first case study we analyze 3D scans of the corneas of the bumblebee \textit{Bombus terrestris}. This analysis shows how persistent homology can be applied to volumetric data and how it can be used to perform a clustering and similarity analysis. Furthermore, we are able to find that the persistent homology, specifically the barcode of $H_{2}$ compared across samples with the bottleneck distance, reinforces the already shown hypothesis in \cite{emily}, namely that the shape of the eyes of \textit{Bombus terrestris} differs between smaller and larger individuals. We also interpret this result as implying that it is the density of the cornea which is a distinguishing factor between differently-sized individuals.

In the second case study we analyze a synthetically generated network made to mimic the micro-circuitry of the striatum. By interpreting the network as a directed graph, we show how persistent homology can be used to compare real-world data given as graphs to control models generated in multiple ways. We reinforce the already established result in \cite{reimann}, that the resulting directed flag complex on the brain network displays a much richer simplicial structure in terms of dimensions and number of simplices compared to control models. Furthermore, we establish that $\beta_{1}$ of the micro-circuitry is much lower than any of the control models. Finally, we see that the distribution of the persistent homology in $H_{1}$ of the micro-circuitry is spread across the entire spectrum of possible degrees, whereas control models only have holes in small intervals of degrees. These observations could act as points of differentiation when it comes to characterizing the striatum.

For some further directions in the first case study, one could extend the methodology to see if $H_{2}$ is always the distinguishing factor within and between different species of insects. However, this would likely require a larger amount of samples.

In the second case study a potential lane of investigation is whether other persistence diagrams of filtrations than the degree based filtration are as unique to the micro-circuitry compared to the other models. Some other filtrations that can be used with the exact same methodology are other measures of importance in a graph, such as the number of shortest path through an edge or the number of neighbors which are neighbors to each other. This is something we wanted to do, but the computational demands together with time restraints made it unfeasible.

Additionally, it could prove fruitful to see whether micro-circuitry in the actual biological striatum displays a similar signature to the synthetic model in terms of persistent homology. Should this be the case, then it strengthens the result of the case study as a signature for the networks of neurons in the striatum. If it is not the case, then persistent homology could perhaps be a parameter to take into account when further calibrating the synthetic model.

%%% Local Variables:
%%% mode: latex
%%% TeX-master: "thesis.tex"
%%% End:
