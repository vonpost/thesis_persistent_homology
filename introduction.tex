\chapter{Introduction}
Although ordinary statistical analysis and machine learning continue to see great success, the ever-changing modern digital landscape suggests there is some value in exploring other avenues in mathematics for understanding data. One such avenue is Topological Data Analysis (TDA), an umbrella term for data analysis achieved through topological methods. While topology and algebraic topology in particular might be seen as something relegated to realms of mathematics, the perhaps most popular technique of TDA, persistent homology, is based on the concept of homology in algebraic topology and has been successfully applied in areas such as neuroscience \cite{reimann}, biology \cite{plants} and material science \cite{moon2019}.

In persistent homology, we approximate a non-trivial topological space, often with a simplicial complex, on the data of interest. From this complex ``holes'' in the resulting space can be found, and these holes are what constitute homology. Now, this approximation is not perfect, and there are multiple ways we can approximate a topological space on a data set. In order to alleviate this imperfection, we define a sequence of complexes ordered by inclusion, all of them valid approximations, and then compute for how many complexes in the sequence a hole persists.

While the high-level idea is not very complicated, the devil is in the details. In order to rigorously define this notion of homology persisting through approximations and keep it flexible for other complexes than simplicial ones, we need to build a robust framework. We do this by defining the general framework of persistence complexes, a sort of complex of complexes, from which we retrieve the holes that persist when going from one complex to another.


Our goal with this thesis is partly to provide an introduction to persistent homology as we would have liked it before we started this journey which is done in Chapters 2 and 3. As such, we have tried to keep a balance between the older material in the field of persistent homology that is foundational and newer material that is more up-to-date. Most of the definitions and results are accompanied by commentary, hopefully providing help along the way. We try to make the theoretical part somewhat self-contained, but some familiarity with linear algebra, category theory, commutative algebra and module theory is needed.

The other part of the thesis is given by Chapter 4 and consists of two case studies. In the first study we analyze the eyes of the bumblebee \textit{Bombus terrestris}, in the second study we analyze a synthetic microcircuit of the striatum in the basal ganglia of the brain. Our goal is to show through these two case studies, despite small, that persistent homology has potential as a tool in the toolbox of data analysis. We have taken care to conduct our analysis in such a way that we highlight how persistent homology enables our approaches.

We owe a lot to a variety of sources cited throughout the thesis. The algebraic framework that we present in Chapter 3 was first developed in \cite{Zomorodian2005}, although it borrows heavily from the more computational view presented in \cite{edelszom}. The articles \cite{vejdemo,skraba} have been of extra importance, as they provide clear overviews of the theory generalized to modules. Our novel contributions are our methodologies in the two case studies, although it we would not be surprised if similar approaches have been conducted before in other domains.


%%% Local Variables:
%%% mode: latex
%%% TeX-master: "thesis.tex"
%%% End:
